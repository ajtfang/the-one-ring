\documentclass[11pt]{article}
\usepackage[utf8]{inputenc}
\usepackage[]{amsmath}
\usepackage{amsthm}
\usepackage[]{upgreek}
\usepackage{xcolor}
\usepackage[]{gensymb}
\usepackage[]{graphicx}
\usepackage[]{nicefrac}
\usepackage{tikz-cd} 
\usepackage{pgfplots}
\usetikzlibrary{decorations.pathmorphing}
\usetikzlibrary{patterns}
\usetikzlibrary{shapes}
\usetikzlibrary{plotmarks}
\usepackage[]{textgreek}
\usepackage[]{url}
\usepackage{xcolor,cancel}
\usepackage{pdfpages}
\usepackage[]{float}
\usepackage{braket}
\usepackage{graphicx, amssymb}
\usepackage{mathrsfs}
\usepackage{geometry}
\usepackage{fancyhdr}
\pagestyle{headings}
\usepackage{sectsty}
\usepackage{slashed}
\usepackage[toc,page]{appendix}
\usepackage{tocloft}
\usepackage{lscape}
\usepackage{bbm}
\usepackage{tcolorbox}
\usepackage[numbers]{natbib}
\newtheorem{theorem}{Theorem}[section]
\newtheorem{corollary}{Corollary}[theorem]
\newtheorem{lemma}[theorem]{Lemma}
\newtheorem{conjecture}{Conjecture}[section]
\newtheorem*{example}{Example}
\newtheorem*{exercise}{Exercise}
\newtheorem*{remark}{Remark}
\newtheorem{prop}{Proposition}
\newtheorem{definition}{Definition}[section]
\renewcommand{\familydefault}{\sfdefault}
\newcommand{\oss}{\underset{o}{\subset}}
\renewcommand{\phi}{\varphi}
 \geometry{
 a4paper,
 left=20mm,
 right=20mm,
 top=20mm,
 bottom=20mm
 }
  \newcommand{\HRule}[1]{\rule{\linewidth}{#1}} 	% Horizontal rule
  \title{	\normalsize \textsc{} 	% Subtitle
		 		\LARGE \textbf{\uppercase{In The Land of Mordor, In The Fires of Mount Doom}}	% Title
		}
		\date{}
 \begin{document}
 \maketitle
 \section{Boosted Schwarzschild Black Strings}
 The boosted Schwarschild black string is a $5D$ solution to the vacuum Einstein equation, 
 \begin{align}
 \mathrm{Ric}(g)=0,
 \end{align}
 which is asymptotically the product of $4D$ Minkowski, $\mathrm{Mink}_4$, and a circle of radius $R$, $\mathbb{S}^1_{R}$. It has metric
 \begin{align}
 g&\doteq-\big(1-[1-D(r)]\cosh^2(\sigma)\big)dt\otimes dt+\frac{1}{D(r)}dr\otimes dr+r^2d\theta\otimes d\theta+r^2\sin^2\theta d\phi\otimes d\phi\\
 &\nonumber\quad+\big(1+[1-D(r)]\sinh^2(\sigma)\big)dz\otimes dz-\big(1-D(r)\big)\cosh(\sigma)\sinh(\sigma)(dt\otimes dz+dz\otimes dt).
 \end{align}
 The coordinate ranges are $t\in[0,\infty)$, $r\in (2M,\infty)$ and $z\in [0,2\pi)$. One can `unboost' the string with the coordinate transformation
 \begin{align}
 \tau&=\cosh\sigma t+\sinh\sigma z,\\
 \zeta&=\sinh\sigma t+\cosh\sigma z.
 \end{align}
 In this case $\zeta,\tau $ have infinite image, so the string direction when unboosted looks not compact?
\section{Singly Spinning Black Rings}
\subsection{Metric}
 The singly spinning black ring solution is a 5D solution to the Einstein vacuum equation, 
\begin{align}
\mathrm{Ric}(g)=0,
\end{align} 
which is asymptotically flat. The metric in coordinates $(t,r,\theta,\phi,\psi)$,
 \iffalse
 \begin{align}
 g\doteq\begin{pmatrix}
 -\frac{f(r)}{h(\theta)}&0&0&0&\frac{\Xi}{rh(\theta)}\Big(\frac{r}{R}-1\Big)\\
 0&\frac{h(\theta)}{(1-\frac{r^2}{R^2})(1+\frac{r\cos\theta}{R})^2p(r)}&0&0&0\\
 0&0&\frac{r^2h(\theta)}{(1+\frac{r\cos\theta}{R})^2q(\theta)}&0&0\\
 0&0&0&\frac{r^2q(\theta)\sin^2\theta}{(1+\frac{r\cos\theta}{R})^2}&0\\
\frac{\Xi}{rh(\theta)}\Big(\frac{r}{R}-1\Big) &0&0&0&\frac{R^2(1-\frac{r^2}{R^2})h(\theta)p(r)}{(1+\frac{r\cos\theta}{R})^2f(r)}-\frac{K^2(1-\frac{r}{R})^2}{r^2f(r)h(\theta)}
 \end{pmatrix}
 \end{align}
 \fi
 \begin{align}
 g&\doteq-\frac{f(r)}{h(\theta)}dt\otimes dt+\frac{h(\theta)}{(1-\frac{r^2}{R^2})(1+\frac{r\cos\theta}{R})^2p(r)}dr\otimes dr+\frac{r^2h(\theta)}{(1+\frac{r\cos\theta}{R})^2q(\theta)}d\theta\otimes d\theta\\
 &\nonumber\quad+\frac{r^2q(\theta)\sin^2\theta}{(1+\frac{r\cos\theta}{R})^2}d\phi\otimes d\phi+\Big(\frac{R^2(1-\frac{r^2}{R^2})h(\theta)p(r)}{(1+\frac{r\cos\theta}{R})^2f(r)}-\frac{\Xi^2(1-\frac{r}{R})^2}{r^2f(r)h(\theta)}\Big)d\psi\otimes d\psi\nonumber\\
 &\nonumber\quad+\frac{\Xi}{rh(\theta)}\Big(\frac{r}{R}-1\Big)\big(dt\otimes d\psi+d\psi\otimes dt\big)\nonumber
 \end{align}
 with the following definitions, 
 \begin{align}
 f(r)&\doteq1-\frac{r_+\cosh^2\sigma}{r}\qquad h(\theta)\doteq1+\frac{r_+\cosh^2\sigma}{R}\cos\theta\\
 p(r)&\doteq1-\frac{r_+}{R},\qquad\qquad q(\theta)\doteq1+\frac{r_+}{R}\cos\theta,
 \end{align}
 and
 \begin{align}
  \Xi&\doteq r_+R\sinh\sigma\cosh\sigma\sqrt{\frac{R+r_+\cosh^2\sigma}{R-r_+\cosh^2\sigma}}.
 \end{align}
 The coordinate ranges are
 \begin{align}
 t\in \mathbb{R}\quad r\in (r_+,R)\quad \theta\in [0,\pi)\quad \phi,\psi\in \Big[0,\frac{2\pi R}{\sqrt{R^2+r_+^2}}\Big)
 \end{align}
 $r=r_+<R$ corresponds to the future event horizon, $\sigma$ is a parameter and $R$ sets the scale of the solution. \\
 
 The inverse metric is: 
 \begin{align}
 g^{-1}&=-\Big(\frac{h(\theta)}{f(r)}-\frac{\Xi^2(R-r)(R+r\cos\theta)^2}{r^2R^4(r+R)p(r)f(r)h(\theta)}\Big)\partial_t\otimes\partial_t+\frac{(R^2-r^2)(R+r\cos\theta)^2p(r)}{R^4h(\theta)}\partial_r\otimes\partial_r\\
 &\nonumber\quad+\frac{(R+r\cos\theta)^2q(\theta)}{r^2R^2h(\theta)}\partial_{\theta}\otimes\partial_{\theta}+\frac{(R+r\cos\theta)^2}{R^2r^2q(\theta)\sin^2\theta}\partial_{\phi}\otimes\partial_{\phi}+\frac{(R+r\cos\theta)^2f(r)}{R^2(R^2-r^2)h(\theta)p(r)}\partial_{\psi}\otimes\partial_{\psi}\nonumber\\
 &\nonumber\quad-\frac{\Xi(R+r\cos\theta)^2}{rR^3(r+R)h(\theta)p(r)}(\partial_{t}\otimes\partial_{\psi}+\partial_{\psi}\otimes\partial_t)\nonumber
 \end{align}
 
  For the spacetime to have no conical singularities on the exterior region we require the `equilibrium' condition:
 \begin{align}
 \cosh^2\sigma=\frac{2R^2}{r_+^2+R^2},\qquad \sinh^2\sigma=\frac{R^2-r_+^2}{r_+^2+R^2}.
 \end{align}
 Thus, 
 \begin{align}
 f(r)&\doteq1-\frac{r_e}{r}\qquad h(\theta)=1+\frac{2Rr_+}{r_+^2+R^2}\cos\theta\\
 p(r)&\doteq1-\frac{r_+}{R}\qquad \qquad q(\theta)\doteq1+\frac{r_+}{R}\cos\theta
 \end{align}
 and
 \begin{align}
 \Xi&\doteq r_e(r_++R)\sqrt{\frac{R+r_+}{2(R-r_+)}}
 \end{align}
 \subsection{Spacetime Regions}
Note that, the vector field $\partial_t$ is null at,
 \begin{align}
 g_{tt}=0\implies r=r_+\cosh^2\sigma
 \end{align}
 which corresponds to the ergosurface. So the ergoregion is given by, 
 \begin{align}
 r\in (r_+,r_e\doteq r_+\cosh^2\sigma)
 \end{align}
 Note the vector field, 
 \begin{align}
 k=\frac{\partial}{\partial t}+\Omega_H\frac{\partial}{\partial\psi} \qquad \Omega_H^2=\frac{R-r_+}{2R^2(R+r_+)}
 \end{align}
is null future event horizon. Note that $\psi$ is has period $\psi\sim \psi+\Delta\psi$ with $\Delta\psi\doteq\frac{2\pi R}{\sqrt{R^2+r_+^2}}$ so $\Omega_H$ is not the true angular velocity of the horizon. Defining $\tilde{\psi}=\frac{2\pi}{\Delta\psi}\psi$ gives
\begin{align}
\partial_{\psi}=\frac{\sqrt{R^2+r_+^2}}{R}\partial_{\tilde{\psi}}. 
\end{align}
So the true angular velocity of the horizon is
\begin{align}
\tilde{\Omega}_H=\frac{1}{R^2}\sqrt{\frac{(R-r_+)(R^2+r_+^2)}{2(R+r_+)}}.
\end{align}


 Now, for $r\in (r_+,r_+\cosh^2\sigma)$, $f(r)<0$ so we can make the coordinate transformation, 
\begin{align}
dv=dt+\frac{\Xi}{r(r+R)\sqrt{-f(r)}p(r)}dr\qquad d\psi=d\chi+\frac{R\sqrt{-f(r)}}{(R^2-r^2)p(r)}dr
\end{align}
which gives the metric, 
\iffalse
\begin{align}
g=\begin{pmatrix}
 -\frac{f(r)}{h(\theta)}&0&0&0&\frac{\Xi}{rh(\theta)}\Big(\frac{r}{R}-1\Big)\\
 0&0&0&0&\frac{R^3h(\theta)}{(R+r\cos\theta)^2\sqrt{-f(r)}}\\
 0&0&\frac{r^2h(\theta)}{(1+\frac{r\cos\theta}{R})^2q(\theta)}&0&0\\
 0&0&0&\frac{r^2q(\theta)\sin^2\theta}{(1+\frac{r\cos\theta}{R})^2}&0\\
\frac{\Xi}{rh(\theta)}\Big(\frac{r}{R}-1\Big) &\frac{R^3h(\theta)}{(R+r\cos\theta)^2\sqrt{-f(r)}}&0&0&\frac{R^2(1-\frac{r^2}{R^2})h(\theta)p(r)}{(1+\frac{r\cos\theta}{R})^2f(r)}-\frac{\Xi^2(1-\frac{r}{R})^2}{r^2f(r)h(\theta)}
 \end{pmatrix}
\end{align}
\fi
\begin{align}
 g&\doteq-\frac{f(r)}{h(\theta)}dv\otimes dv+\frac{r^2h(\theta)}{(1+\frac{r\cos\theta}{R})^2q(\theta)}d\theta\otimes d\theta+\frac{r^2q(\theta)\sin^2\theta}{(1+\frac{r\cos\theta}{R})^2}d\phi\otimes d\phi\\
 &\nonumber\quad+\Big(\frac{R^2(1-\frac{r^2}{R^2})h(\theta)p(r)}{(1+\frac{r\cos\theta}{R})^2f(r)}-\frac{\Xi^2(1-\frac{r}{R})^2}{r^2f(r)h(\theta)}\Big)d\chi\otimes d\chi\nonumber\\
 &\nonumber\quad+\frac{\Xi}{rh(\theta)}\Big(\frac{r}{R}-1\Big)\big(dv\otimes d\chi+d\chi\otimes dv\big)+\frac{R^3h(\theta)}{(R+r\cos\theta)^2\sqrt{-f(r)}}(dr\otimes d\chi+d\chi\otimes dr)\nonumber
 \end{align}
Then one has the following for $k$, 
\begin{align}
k=\partial_v+\Omega_H\partial_{\chi}.
\end{align}
Mapping $k$ to a one-form, 
\begin{align}
k_{\flat}&=\Big[\frac{\Xi\Omega_H}{rh(\theta)}\Big(\frac{r}{R}-1\Big)-\frac{f(r)}{h(\theta)}\Big]dv+\Omega_H\frac{R^3h(\theta)}{(R+r\cos\theta)^2\sqrt{-f(r)}}dr\\
&\nonumber\quad+\Big[\frac{\Xi}{rh(\theta)}\Big(\frac{r}{R}-1\Big)+\Omega_H\Big(\frac{R^2(1-\frac{r^2}{R^2})h(\theta)p(r)}{(1+\frac{r\cos\theta}{R})^2f(r)}-\frac{\Xi^2(1-\frac{r}{R})^2}{r^2f(r)h(\theta)}\Big)\Big]d\chi
\end{align}
which at the future event horizon for a balanced ring gives, 
\begin{align}
k_{\flat}|_{r=r_+}=\frac{R^2}{\sqrt{2(R^2+r_+^2)}(R+r_+)}\frac{(R^2+r^2_0+2Rr_+\cos\theta)}{(R+r_+\cos\theta)^2}dr.
\end{align}
Note that,
\begin{align}
g(k,k)=-\frac{f(r)}{h(\theta)}+2\Xi\Omega_H\frac{\big(\frac{r}{R}-1\big)}{rh(\theta)}+\Omega_H^2\frac{1}{f(r)h(\theta)}\Big(\frac{R^2(R^2-r^2)h(\theta)^2p(r)}{(R+r\cos\theta)^2}-\frac{(\frac{r}{R}-1)^2\Xi^2}{r^2}\Big).
\end{align}
To find the surface gravity, $\kappa$, of a balanced ring we consider $d(g(k,k))|_{r=r_+}=-2\kappa k|_{r=r_+}$. Evaluating, one finds, 
\begin{align}
\kappa=\frac{(R-r_+)\sqrt{(R^2+r_+^2)}}{2\sqrt{2}R^2r_+}.
\end{align}

\end{document}