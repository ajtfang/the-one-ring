\documentclass{article}

\usepackage{Ring}
\usepackage[sortcites,
backend=biber,
date=year,
style=numeric-comp,
doi=false,
isbn=false,
url=false,
eprint=true]{biblatex}
%% Bibliography Customization
% \DeclareNameAlias{author}{family-given}
% \AtEveryBibitem{
%   \clearfield{month}
%   \clearfield{url}
%   \clearfield{urlyear}
%   \clearfield{urlmonth} 
%   \ifentrytype{online}{
%     \clearfield{year}}
%   {
%     \clearfield{eprint}}
% }
% \renewbibmacro*{volume+number+eid}{%
%   \printfield{volume}%
% %  \setunit*{\adddot}% DELETED
%   \setunit*{\addnbspace}% NEW (optional); there's also \addnbthinspace
%   \printfield{number}%
%   \setunit{\addcomma\space}%
%   \printfield{eid}}
% \DeclareFieldFormat[article]{volume}{\textbf{#1}}
% \DeclareFieldFormat[article]{number}{\mkbibparens{#1}}
% \DeclareFieldFormat*{title}{\textit{#1}}
% \DeclareFieldFormat{journaltitle}{#1\isdot}
% \renewbibmacro{in:}{}
% \addbibresource{Ring.bib}
\title{The One Ring}
\date{}
\begin{document}
\maketitle

\begin{enumerate}
\item The first thing would be to set up the spectral analysis. I
  suspect that this should be similar to what can be done for
  Schwarzschild, although I'm not sure whether we would need Gevray
  spaces. Ideally, the only tools you would need are redshift
  estimates and the $\p_t$ estimate. (I don't think this is necessary
  since we are only interested in studying growing modes.)
\item With the spectral theory, we want to see ideally that the black
  string mode is isolated (it is part of the point spectrum). This
  could be seen by proving some Fredholm property of the resolvent (or
  by basic spectral theory if the problem allows for that).
\item The next step is to study the perturbation theory, here, I think
  the main difficulty will center around the fact that we need to make
  sure that perturbing a true mode generates a true mode. We know that
  the harmonic gauge constraint is propagated by a wave-type operator,
  so the modes there can be perturbed in a good way. For the pure
  gauge modes, we should see if theres a way to use the $W_{tztz}$
  condition.
\item Also need the expansion of the black ring around the black
  string as a true perturbation.
\end{enumerate}

\section{Setup for spectral analysis}

\subsection{Estimates}

\subsubsection{Redshift}

\subsubsection{Killing energy estimate}

\subsection{Proving a Fredholm alternative}

\section{Perturbation theory}

\subsection{Perturbation theory for principally scalar wave operators}

\subsection{Mode categorization}

There are three classes of modes we are interested in.
\begin{definition}
  We call a mode solution $h = e^{\mu t}u(x)$ a \emph{pure gauge mode
    solution}\footnote{This differs from the definition we use in
    stability of Kerr-de Sitter where we also consider modes that are
    linearized changes of the black hole parameters. We should
    definitely check if we also need to consider these.} if there exists a smooth vectorfield
  $\vartheta$ such that
  \begin{equation*}
    h = \Lie_{\vartheta}g.
  \end{equation*}
\end{definition}

\begin{definition}
  We call a mode solution $h = e^{\mu t}u(x)$ a \emph{pure constraint
    mode solution} if $h$ does not satisfy the linearized harmonic
  coordinate constraints.\footnote{Probably also need to consider the
    spherical gauge that is used.}
\end{definition}

\begin{definition}
  We call a mode solutions a \emph{true mode solution} if it is
  neither a pure gauge mode nor a pure constraint mode. 
\end{definition}

For the sake of this paper, we are mainly interested in the true mode
solutions. The perturbation argument should proceed roughly as follows:
\begin{enumerate}
\item We will perform mode perturbation on two separate operators,
  those being  
  \begin{equation*}
    \LinEinstein = D_g\Ric - \widetilde{\nabla_g\otimes} D_g\Constraint(g, g^0),\qquad
    \widetilde{\ConstraintPropagationOp},
  \end{equation*}
  the constraint-damped linearized Einstein operator in harmonic gauge
  and the constraint-damped constraint propagation operator for
  harmonic gauge respectively. Both of these operators are principally
  the scalar wave operator and as such, should be amenable to mode
  perturbation \emph{in the upper half plane} in a similar way to
  Kerr-de Sitter. The main general result that we will use from the general
  perturbation theory is that for small perturbations, the total
  number of modes is equal (at an eigenvalue). 
  
\item There is an exact correspondence between the pure gauge mode
  solutions to the linearized Einstein operator and the constraint
  propagation operator. This gives us a way of counting the pure gauge
  modes in the perturbation since the constraint propagation operator
  is principally hyperbolic.
  
\item If we use constraint damping, then we eliminate any growing
  modes associated to constraint violations. This would leave us in
  the setting where there are only pure gauge modes or true
  modes. Since we have a way of counting pure gauge modes, this shows
  us that we must the same number of true modes in the perturbed case
  as well.
\end{enumerate}

\section{Linearized Einstein vacuum equations}
\label{sec:lin-eve}

In this section, we introduce the linearized Einstein equations in the
notation that we will use.  Directly linearizing \eqref{eq:EVE:Full}
around $g$ yields the ungauged linearized Einstein equation
\begin{equation}
  \label{eq:linearized-EVE-ungauged}
  D_g\Ric(h) = 0. 
\end{equation}
Given admissible initial data $(\Sigma_0, \InducedMetric_0, k_0)$ for $g$,
we define the \emph{linearized constraint equation} as the
linearization of \eqref{eq:EVE:constraint-eqns} around
$(\InducedMetric_0, k_0)$ in terms of the linearized metric
$\InducedMetric'$ and the linearized second fundamental form
$k'$. % , and can be expressed as
% \begin{equation}
%   \label{eq:linearized-constraint-eqns}
%   \begin{split}
%     -\nabla^i\nabla_i \tensor[]{{\InducedMetric'}}{^j_j} + \nabla^i\nabla^j{\InducedMetric'}_{ij} +\Lambda \Trace_{g}{\InducedMetric'} &= 0,\\
%     \nabla_i\left(k'^{ij}-\Trace_{g} k'(g)^{ij}\right) &=0,  
%   \end{split}
% \end{equation}
% where the covariant derivatives are those of $g$.
An initial data triplet $(\Sigma_0, \InducedMetric', k')$ linearized
around $(\InducedMetric_b, k_b)$ is an \textit{admissible} initial
data triplet for Einstein equations linearized around $g$ if
$(\InducedMetric', k')$ satisfy the linearized constraint equations.
Linearizing the gauged Einstein equations in \eqref{eq:EVE:gauged-eq},
we have the linearized gauged Einstein equations
\begin{equation}
  \label{eq:linearized-gauged-EVE}
  D_{g}(\Ric - \Lambda)({h}) - \nabla_{g}\otimes D_{g}\Constraint(g, g^0)(h) = 0. 
\end{equation}

\begin{definition}
  Define the \emph{linearized gauge constraint} \begin{equation*}
    \begin{split}
      \Constraint_{g}{h}:={}&D_{g}\Constraint(g+{h}, g)({h})\\
      ={}& -\nabla_{g}\cdot\TraceReversal_{g}{h}.
    \end{split}    
  \end{equation*}
\end{definition}
We have the following linearized equivalent of Lemma
\ref{lemma:EVE:nonlinear-constraint-prop}.
\begin{lemma}
  \label{lemma:EVE:linearized-constraint-prop}
  Let ${h}$ solve \eqref{eq:linearized-gauged-EVE}. Then ${h}$
  also satisfies
  \begin{equation*}
    \ConstraintPropagationOp_{g}(\Constraint_{g}{h}) = 0, \qquad
    \ConstraintPropagationOp_{g}\psi = \VectorWaveOp_g\psi, 
  \end{equation*}
  where $\VectorWaveOp_g = \nabla^\alpha \nabla_\alpha $ denotes
  the wave operator acting on 1-tensors. 
\end{lemma}
\begin{proof}
  The lemma follows directly by applying the twice-contracted
  linearized second Bianchi identity to the gauged linearized Einstein
  equation.
\end{proof}
\begin{remark}
  From Lemma \ref{lemma:EVE:linearized-constraint-prop}, it is clear
  that if $(\evalAt*{\Constraint_{g}{h}}_{\Sigma_0}, \evalAt*{\LieDerivative_{\KillT}\Constraint_{g}{h}}_{\Sigma_0}) = (0, 0)$, then
  $\Constraint_{g}{h} = 0$ uniformly.  
\end{remark}

Finally, we remark that any solution ${h}$ to the ungauged linearized
Einstein's equations \eqref{eq:linearized-EVE-ungauged} can be put
into the linearized gauge $\Constraint_{g}({h}) = 0$ by finding some
infinitesimal diffeomorphism
$\nabla_{g}\otimes \omega$\footnote{Observe that in terms of the Lie
  derivative, we have that
  \begin{equation*}
    \nabla_{g}\otimes \omega = \frac{1}{2}\LieDerivative_{\omega^\sharp}g.
  \end{equation*}
}
such that
\begin{equation}
  \Constraint_{g}({h} + \nabla_{g}\otimes \omega) = 0,
\end{equation}
as general covariance implies that
\begin{equation*}
  D_{g}\Ric(\nabla_{g}\otimes \omega) = 0
\end{equation*}
for any one-form $\omega\in C^\infty(\mathcal{M},
T^*\mathcal{M})$. This is equivalent to finding some
$\omega$ such that
\begin{equation}
  \label{eq:linearized-gauge-fixer}
  \Box_{g}^\Upsilon\omega = 2\Constraint_{g}({h}),\qquad
  \Box_{g}^{\Upsilon} = -2\Constraint_{g}\circ \nabla_{g}\otimes ,
\end{equation}
which is principally $\VectorWaveOp_g$, and in fact, in our case we
can calculate that
\begin{equation*}
  \Box_{g}^\Upsilon = \VectorWaveOp_g. 
\end{equation*}
Solving for $\omega$ satisfying \eqref{eq:linearized-gauge-fixer} with
Cauchy data
$\left.(\omega, \LieDerivative_{\KillT}\omega)\right\vert_{\Sigma_0}=
0$ then ensures that $h +\nabla_g\otimes \omega$ has the same initial
data as $h$.


\subsection{Constraint damping}
\label{sec:constraint-damping}

The goal of constraint damping is to modify the operator
\begin{equation*}
  \nabla_g\otimes
\end{equation*}
by some lower-order terms so that the quasinormal modes of the
constraint propagation operator $\ConstraintPropagationOp_g$ lie
strictly in the exponentially decaying half-space. This will force any
exponentially growing modes of the constraint-damped linearized
Einstein operator to satisfy the harmonic gauge constraints.

We remark that since this is a lower-order modification, this will not
affect the principal hyperbolic nature of the linearized Einstein
operator in harmonic gauge.

We will combine two natural lower-order modifications of $\nabla_g\otimes$:
\begin{enumerate}
\item Conjugation of $\delta_g\otimes$ with an exponential,
  \begin{equation*}
    e^{-\gamma t}\nabla_g\otimes e^{\gamma t}.
  \end{equation*}
\item Using the conformally weighted
  \begin{equation*}
    \nabla_{e^{-2\gamma t}g}\otimes.
  \end{equation*}
\end{enumerate}
Combining these two in a general linear combination, we have the
following modification of $\nabla_g\otimes$:
\begin{equation*}
  \widetilde{\nabla_g\otimes}\omega
  := \nabla_G\otimes\omega
  + \gamma_1 dt\cdot \omega
  - \gamma_2 \left(i_{\nabla_t}\omega\right)g, \qquad
  \gamma_1,\gamma_2 \in \Real.
\end{equation*}
The main result is then as follows.
\begin{theorem}
  There exist parameters $\gamma_1,\gamma_2>0$ and a constant
  $\alpha>0$ such that all quasinormal modes $\sigma$ of the
  constraint-damped constraint propagation operator
  $\widetilde{\ConstraintPropagationOp}_g$ satisfy
  $\Im\sigma<-\alpha$.
\end{theorem}


\end{document}

%%% Local Variables:
%%% mode: latex
%%% TeX-master: t
%%% End:
