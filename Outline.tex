\documentclass{article}

\usepackage{Ring}
\usepackage[sortcites,
backend=biber,
date=year,
style=numeric-comp,
doi=false,
isbn=false,
url=false,
eprint=true]{biblatex}
%% Bibliography Customization
% \DeclareNameAlias{author}{family-given}
% \AtEveryBibitem{
%   \clearfield{month}
%   \clearfield{url}
%   \clearfield{urlyear}
%   \clearfield{urlmonth} 
%   \ifentrytype{online}{
%     \clearfield{year}}
%   {
%     \clearfield{eprint}}
% }
% \renewbibmacro*{volume+number+eid}{%
%   \printfield{volume}%
% %  \setunit*{\adddot}% DELETED
%   \setunit*{\addnbspace}% NEW (optional); there's also \addnbthinspace
%   \printfield{number}%
%   \setunit{\addcomma\space}%
%   \printfield{eid}}
% \DeclareFieldFormat[article]{volume}{\textbf{#1}}
% \DeclareFieldFormat[article]{number}{\mkbibparens{#1}}
% \DeclareFieldFormat*{title}{\textit{#1}}
% \DeclareFieldFormat{journaltitle}{#1\isdot}
% \renewbibmacro{in:}{}
% \addbibresource{Ring.bib}
\title{The One Ring}
\date{}
\begin{document}
\maketitle

\begin{enumerate}
\item The first thing would be to set up the spectral analysis. I
  suspect that this should be similar to what can be done for
  Schwarzschild, although I'm not sure whether we would need Gevray
  spaces. Ideally, the only tools you would need are redshift
  estimates and the $\p_t$ estimate. (I don't think this is necessary
  since we are only interested in studying growing modes.)
\item With the spectral theory, we want to see ideally that the black
  string mode is isolated (it is part of the point spectrum). This
  could be seen by proving some Fredholm property of the resolvent (or
  by basic spectral theory if the problem allows for that).
\item The next step is to study the perturbation theory, here, I think
  the main difficulty will center around the fact that we need to make
  sure that perturbing a true mode generates a true mode. We know that
  the harmonic gauge constraint is propagated by a wave-type operator,
  so the modes there can be perturbed in a good way. For the pure
  gauge modes, we should see if theres a way to use the $W_{tztz}$
  condition.
\item Also need the expansion of the black ring around the black
  string as a true perturbation.
\end{enumerate}

\section{Setup for spectral analysis}

\subsection{Estimates}

\subsubsection{Redshift}

\subsubsection{Killing energy estimate}

\subsection{Proving a Fredholm alternative}

\section{Perturbation theory}

\subsection{Perturbation theory for principally scalar wave operators}

\subsection{Mode categorization}

There are three classes of modes we are interested in.
\begin{definition}
  We call a mode solution $h = e^{\mu t}u(x)$ a \emph{pure gauge mode
    solution}\footnote{This differs from the definition we use in
    stability of Kerr-de Sitter where we also consider modes that are
    linearized changes of the black hole parameters. We should
    definitely check if we also need to consider these.} if there exists a smooth vectorfield
  $\vartheta$ such that
  \begin{equation*}
    h = \Lie_{\vartheta}g.
  \end{equation*}
\end{definition}

\begin{definition}
  We call a mode solution $h = e^{\mu t}u(x)$ a \emph{pure constraint
    mode solution} if $h$ does not satisfy the linearized harmonic
  coordinate constraints.\footnote{Probably also need to consider the
    spherical gauge that is used.}
\end{definition}

\begin{definition}
  We call a mode solutions a \emph{true mode solution} if it is
  neither a pure gauge mode nor a pure constraint mode. 
\end{definition}

For the sake of this paper, we are mainly interested in the true mode solutions. 

\end{document}

%%% Local Variables:
%%% mode: latex
%%% TeX-master: t
%%% End:
